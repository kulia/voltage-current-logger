\documentclass[preview]{standalone}
\usepackage{circuitikz}
\begin{document}

<<<<<<< HEAD
\begin{circuitikz}[american] \draw

	  (0,0) node [transformer core](T){}  % reminded by @PaulGessler, thanks.
      (T.A1) node[above] {}
      (T.A2) node[below] {}
      (T.B1) node[above] {} 
      (T.B2) node[below] {}
      (T.base) node{};
      
\draw (T.A1) to[open,v<={$230V_{rms}$}](T.A2);
\draw (T.B2) to[open,v>=$6 V$](T.B1);

\draw (T.B1) -- (2,0);
\draw (2,0) -- (2,1);
\draw (2,1) -- (3,1);

\draw (T.B2) -- (2,-2.1);
\draw (2,-2.1) -- (2,-3);
\draw (2,-3) -- (3,-3);

\draw 	(3,1) to[R, l=$R_1$, -*] (3,-1)
		(3,-1) to[R, l=$R_2$] (3,-3)
;

\draw (3,-1) to[short, -o] (4,-1);
\draw (4,-1) node[right] {$v_{out}$}

\end{circuitikz}
=======
<<<<<<< HEAD
\begin{circuitikz} \draw

(0,0) to[sinusoidal voltage source, o-o,l=$v_{in}$] (0,2)
(0,2) to[R, -*, l=$1.2\text{ K\Omega}$] 				(2,2)
(0,0) node[ground]

(2,2) to[R, l=$4.6\text{ K\Omega}$]			(2, 0)
(2,0) node[ground]

(2,2) to[short, -o] 								(3, 2)
(3,2) node[right] {$v_{out}$}

(2,2) -- (2,4)
(2,4) to[R, -,l=$1.2\text{ K\Omega}$] (6,4)
(6,0) to[american voltage source, l=$5\ V$] (6,4)
(6,0) node[ground]

;
\end{circuitikz}
=======
<<<<<<< HEAD
\begin{figure}
	\begin{circuitikz}[american, decoration={coil}]
		\draw(-1.5,-2)to [short,i=$I_p$](1.5,2);
		\fill[gray, even odd rule] (0,0) circle[radius=1cm] circle[radius=1.5cm];
		\draw (0,0) node [gray, left=1.5cm, above=1.5cm]{magnetic  core};
		\begin{scope}
		\draw[decorate, decoration={aspect=0.4, segment length=3mm, amplitude=5mm}]
		(1.25,-0.8) coordinate (a) --node[midway,right=0.5cm]{} + (0,2) coordinate (b);
		\end{scope}
		% right circuit
		\draw(4,1.2)coordinate(e1) to[R=R] (4,-0.8)coordinate (e2) ;

		\draw(6,1.2)coordinate(e3) to[open, o-o,l=$v_{out}$] (6,-0.8)coordinate (e4) ;
		
		\draw (a) -- (e2)--(e4);
		\draw (b) -- (e1)--(e3);
		\draw (a) to [short,i=$I_s$](e2) ;
		\draw(-1.5,-2)--(0.5,0.66);


	\end{circuitikz}
% \caption{Ammeter uses the method current transformer}
\end{figure}
=======
\begin{circuitikz} \draw


(0,0) node[op amp](opamp){}

(opamp.up) -- (-0.1,1) node[above]{$V_{cc}$}
(opamp.down) -- (-0.1,-1) node[below]{$-V_{cc}$}
(opamp.+) -- (-1.2,-2){}
(opamp.-) to[R, *-,l=$R_2$] (-3, 0.5){}
(opamp.-) -- (-1.2,2){}

(-1.2,-2)node[ground] {}
(-3, 0.5) to[sinusoidal voltage source, l=$V_{in}$] (-3, -2) {}
(-3, -2)  node[ground]

(-1.2,2) to[R, l=$R_1$] 	(1.2, 2)
(1.2, 2) 	--	(1.2,0)


(1.2,0)	to[short, -o] (2,0)
(2,0) node[right] {$v_{adc}$}
% to[open, o-o,l=$v_{out}$] (1.2, -2)

;
\end{circuitikz}
>>>>>>> aa21b8c788d50b4978bd7ca6070755dab3233187
>>>>>>> ca6918ebf5834669920be06a091fa6a35d91225c
>>>>>>> 6aa4aa7998c2d6353386536d7e03c5f556acd87b

\end{document}